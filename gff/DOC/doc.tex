\documentclass{article}

\usepackage{skd}
%%%%%%%%%%%%%%%%%%%%%%%%%%%%%%%%%%%%%%%%%%%%%%%%%%%%%%%%%%%%%%%%%%%%%%
\newcommand{\gff}{{\sf gff }}
%%%%%%%%%%%%%%%%%%%%%%%%%%%%%%%%%%%%%%%%%%%%%%%%%%%%%%%%%%%%%%%%%%%%%%
\begin{document}

\title{\textsf{\textbf{gff: A Tool for Computing FIRST and FOLLOW Sets}}}

\author{{Saumya Debray}\\
      {\em Department of Computer Science}\\
      {\em University of Arizona}\\
      {\em Tucson, AZ 85721.}
  }


\date{~}

\maketitle
%%%%%%%%%%%%%%%%%%%%%%%%%%%%%%%%%%%%%%%%%%%%%%%%%%%%%%%%%%%%%%%%%%%%%%%%%%%%%%%%

\section{Description}

\gff is a simple tool that takes as input a contex free grammar, and
produces as output the FIRST and FOLLOW sets for the nonterminals of the
grammar.  It's a pretty bare-bones no frills tool, with only rudimentary
error checking and handling.

As a simple example, consider the CFG
%
\begin{quote}
$S \longrightarrow (~ S ~) ~\mid~ \varepsilon$
\end{quote}
%
To compute the FIRST and FOLLOW sets for this grammar, we first create a
file---say, {\tt ex.in}---specifying the grammar:
%
\begin{quote}
{\tt
ex.in:\\
\begin{tabular}{|l|}\hline
\begin{minipage}{3.5in}
\begin{tabbing}
S \= \kill
\%\%\\
S : '(' S ')'\\
\> $|$ /* epsilon */~~~~~\\
\> ;
\end{tabbing}
\end{minipage}
\\ \hline
\end{tabular}
}
\end{quote}
%
The first line of this file, `{\tt \%\%}', is a delimiter that separates 
some optional declarations (whicch come before the {\tt \%\%} delimiter---in
this case, there are no declarations) from the grammar productions.  The
syntax of the input files is discussed below.\footnote{People acquainted
with the tools {\tt yacc} or {\tt bison} will see strong similarities
between the syntax of \gff input files and those for {\tt yacc}/{\tt bison}.
This is not an accident.}

\gff is invoked on this file as follows (with `{\tt \%}' being
the shell prompt):
%
\begin{quote}
{\tt \% gff ex.in}
\end{quote}
%
and produces the following output:
%
\begin{quote}
\begin{verbatim}
FIRST sets:
  S:   '(' <epsilon> 

FOLLOW sets:
  S:   <EOF> ')' 
\end{verbatim}
\end{quote}

\section{\gff Usage}

\gff takes a single argument, the name of a file specifying a context free
grammar.  It produces as output, on {\tt stdout}, the FIRST and FOLLOW sets
of the nonterminals of that grammar.

\section{Input Syntax}

\subsection{File Structure}

An input file consists of two parts: a {\em declarations} part and a
{\em productions} part.  The two parts are separated by the delimiter
`{\tt \%\%}' on a line by itself.  The file therefore has the following
structure: 
%
\begin{quote}
{\tt
\begin{tabular}{|l|}\hline
\begin{minipage}{3.5in}
$\left\}
   \begin{array}{ll}
	{~}\\
	{\mbox{\it declarations}}\\
	{~}\\ 
   \end{array}
 \right.$\\
\%\%\\
$\left\}
   \begin{array}{ll}
	{~}\\
	{\mbox{\it grammar rules}}\\
	{~}\\ 
   \end{array}
 \right.$\\
\end{minipage}
\\ \hline
\end{tabular}
}
\end{quote}
%
Both the declarations and the grammar rules are optional, but the 
delimiter between them is required.  Thus, the
shortest legal \gff input would be the file consisting simply of the
`{\tt \%\%}' delimiter.

\subsection{Comments}

Comments are as in C: a character sequence (possibly spanning multiple
lines) starting with `{\tt /*}' and terminated by `{\tt */}', with
no occurrences of `{\tt */}' in between.

Comments can appear anywhere in the input file except for the line
containing the `{\tt \%\%}' delimiter.

\subsection{Identifiers}\label{sec:ID}

Identifiers are used to name nonterminals and, possibly, terminals.
An identifier is as in C, and consists of a letter followed by zero or more
letters, digits, and underscores.

Examples: {\tt ID}, {\tt IntegerConst}, {\tt expression}, {\tt stmt\_list}

\subsection{Declarations}

The optional declarations section specifies identifiers that are token
names, as well as the start symbol of the grammar.

\subsubsection{Tokens}\label{sec:tokens}

Tokens are declared using the `{\tt \%token}' keyword.  Tokens that
are given names, e.g., ``IDENT'' (identifier) or ``INTCONST'' (integer
constant), must be declared as tokens; ``character constant'' tokens
(a single character preceded and followed by a single quote)
e.g., {\tt '+'}, {\tt '('}, need not be explicitly declared.

The declarations section can contain zero or more
token declarations.  Each such declaration is of the form
%
\begin{quote}
{\tt \%token} {\em id\_list}
\end{quote}
%
where {\em id\_list} is a list of token names
separated by whitespace.  Each token name is an identifier.
As an example, suppose that the identifiers we want to declare are: ID,
INTCONST, STRINGCONST, and CHARCONST.  The following are all acceptable ways
to do this:
%
\begin{enumerate}
\item
A single `{\tt \%token}' declaration with a list of identifiers:
\begin{quote}
\begin{tabular}{|l|}\hline
\begin{minipage}{3.5in}
\ \\
{\tt \%token ID INTCONST STRINGCONST CHARCONST}
\ \\
\end{minipage}
\\ \hline
\end{tabular}
\end{quote}

\item
Multiple `{\tt \%token}' declarations, each with a list of identifiers:
\begin{quote}
\begin{tabular}{|l|}\hline
\begin{minipage}{3.5in}
\ \\
{\tt \%token ID INTCONST}\\
{\tt \%token STRINGCONST CHARCONST}
\ \\
\end{minipage}
\\ \hline
\end{tabular}
\end{quote}

\item
Multiple `{\tt \%token}' declarations, each with a single identifier:
\begin{quote}
\begin{tabular}{|l|}\hline
\begin{minipage}{3.5in}
\ \\
{\tt \%token ID}\\
{\tt \%token INTCONST}\\
{\tt \%token STRINGCONST}\\
{\tt \%token CHARCONST}
\ \\
\end{minipage}
\\ \hline
\end{tabular}
\end{quote}
\end{enumerate}

\subsubsection{The Start Symbol}

The start symbol of the grammar (which affects FOLLOW sets) can be specified
explicitly, in the declarations portion of the input file, using the 
`{\tt \%start}' keyword: 
%
\begin{quote}
{\tt \%start} $\quad$ {\em id}
\end{quote}
%
where {\em id} is an identifier that is the name of a nonterminal.  

Start symbol specifications are optional: if no start symbol is specified,
the first nonterminal listed in the {\em grammar rules} section of the
input file will be taken to be the start symbol.  A grammar should not
specify more than one start symbol: if multiple `{\tt \%start}' declarations
are encountered in an input file, the first one is used.

As an example, suppose we have a grammar with nonterminals
$X$ and $Y$, where $Y$ is the start symbol.  The following would explicitly
declare $Y$ to be the start symbol:
%
\begin{quote}
\begin{tabular}{|l|}\hline
\begin{minipage}{3.5in}
\ \\
{\tt \%token \ldots}\\
{\tt \%start $Y$}\\
{\tt \%\%}\\
{\em \ldots grammar rules for $X$ \ldots}\\
{\em \ldots grammar rules for $Y$ \ldots}\\
\end{minipage}
\\ \hline
\end{tabular}
\end{quote}
%
The start symbol could also be declared implicitly, as follows, where
$Y$ is taken as the start symbol because it is the first nonterminal whose
rules are listed in the grammar rules portion of the input:
%
\begin{quote}
\begin{tabular}{|l|}\hline
\begin{minipage}{3.5in}
\ \\
{\tt \%token \ldots}\\
{\tt \%\%}\\
{\em \ldots grammar rules for $Y$ \ldots}\\
{\em \ldots grammar rules for $X$ \ldots}\\
\end{minipage}
\\ \hline
\end{tabular}
\end{quote}

\subsection{Grammar Rules}

The grammar rules section of the input file consists of a list of
nonterminals and their productions;
%
\begin{quote}
\begin{tabular}{|l|}\hline
\begin{minipage}{3.5in}
\ \\
\ldots\\
{\tt \%\%}\\
{\em ${\it nonterminal}_1$'s productions}\\
\ldots\\
{\em ${\it nonterminal}_n$'s productions}\\
\end{minipage}
\\ \hline
\end{tabular}
\end{quote}

\subsubsection{Nonterminals}

Nonterminal names are identifiers (see Section \ref{sec:ID}).  Any
identifier that has not been declared to be a token (see Section
\ref{sec:tokens}) is taken to be a nonterminal.

\subsubsection{Productions}

The productions for a nonterminal are specified as follows.  Given a
nonterminal $X$ with productions
%
\begin{quote}
$X \rightarrow \alpha_1$\\
$X \rightarrow \alpha_2$\\
$\cdots$\\
$X \rightarrow \alpha_n$
\end{quote}
%
the productions would be written as: the nonterminal $X$; then
a colon delimiter; then a list of the right hand sides of its productions
separated by `$\mid$'; and finally ending with a semicolon:
%
\begin{quote}
$X : \alpha_1 ~\mid~ \alpha_2 ~\mid~ \cdots ~\mid~ \alpha_n ~;$
\end{quote}
%
There is exactly one such specification for each nonterminal in the grammar.
Thus, for a grammar with nonterminals $X_1, \ldots, X_k$, the grammar rules
$X_1, \ldots, X_k$, where nonterminal $X_i$ has productions
%
\begin{quote}
$X_i \rightarrow \alpha_{i1} ~\mid~ \cdots \alpha_{in_i}$
\end{quote}
%
the productions would be specified as
%
\begin{quote}
\begin{tabular}{|l|}\hline
\begin{minipage}{3.5in}
\ \\
{\tt \ldots}\\
{\tt \%\%}\\
{\tt $X_1$ : $\alpha_{11} ~\mid~ \cdots ~\mid~ \alpha_{1n_1}$ ;}\\
{\tt \ldots}\\
{\tt $X_k$ : $\alpha_{k1} ~\mid~ \cdots ~\mid~ \alpha_{kn_k}$ ;}\\
\end{minipage}
\\ \hline
\end{tabular}
\end{quote}
%
If one of the right hand sides is $\varepsilon$, it is written simply
as expected, e.g.: the productions
%
\begin{quote}
$S \rightarrow ( ~ S ~) ~\mid~ \varepsilon$
\end{quote}
%
would be written as:
%
\begin{quote}
\begin{tabular}{|l|}\hline
\begin{minipage}{3.5in}
\ \\
{\tt \ldots}\\
{\tt \%\%}\\
{\tt S : '(' S ')'}\\
{\tt \verb+  +$|$ /* epsilon */}\\
{\tt ;}\\
\end{minipage}
\\ \hline
\end{tabular}
\end{quote}

\section{An Example}

Consider the context free grammar $G = (V, T, P, E)$, where:
%
\begin{itemize}
\item[--]
$V = \{E, E_1, T, T_1, F\}$ is the set of nonterminals;
\item[--]
$T = \{{\bf id}, {\bf intcon}, +, -, *, /, (, )\}$ is the set of terminals;
\item[--]
the start symbol is $E$; and
\item[--]
the set of productions $P$ consists of the following:
%
\begin{quote}
$E \rightarrow T ~ E_1$\\
$E_1 \rightarrow + ~T ~E_1 ~\mid~ - ~T ~E_1 ~\mid~ \varepsilon$\\
$T \rightarrow F ~ T_1$\\
$T_1 \rightarrow * ~F ~T_1 ~\mid~ / ~F ~T_1 ~\mid~ \varepsilon$\\
$F \rightarrow {\bf id} ~\mid~ {\bf intcon}  ~\mid~ ( ~E~ )$
\end{quote}
\end{itemize}
%
A \gff input file specifying this grammar would be:
%
\begin{quote}
\begin{tabular}{|l|}\hline
\begin{minipage}{3.5in}
\begin{verbatim}

%token ID INTCON
%start E
%%
E : T E1
;

E1 : '+' T E1
   | '-' T E1
   |  /* epsilon */
;

T : F T1
;

T1 : '*' F T1
   | '/' F T1
   |  /* epsilon */
;

F : ID
  | INTCON
  | '(' E ')'
;

\end{verbatim}
\end{minipage}
\\ \hline
\end{tabular}
\end{quote}
%
\section*{Some style comments}
It simplifies life and improves readability (especially
if you're going to be making changes to the grammar) to have each
right-hand-side of a production listed on a separate line, as in the example
above.  Also, indicating an epsilon-production via an explicit comment,
as shown above, helps readibility.



\end{document}
